\documentclass{article}
%
\usepackage[mathcal,mathbf]{euler}
\usepackage{theorem,amsmath,enumerate,fancyhdr,amssymb,amsfonts}
\usepackage[pdftex]{graphics}

\usepackage{myDefs}

\begin{document}

\pagestyle{fancy}
\lhead{{\bf Assignment 1}
\\{\bf Author: }{Yuan Qu} } 
\rhead{{\bf Date: }01/31/2017} 

\section*{Question 1}{
    \subsubsection*{a.}{
        First, prove that \(\emptyset \in \mathit{S}\).\\
        Obviously, we have \(\emptyset \in \mathit{E}=\{\mathrm{1,2,...,n}\}\), and \(\mathit{a}(\emptyset)=\sum_{\mathit{j} \in \emptyset}=\mathrm{0} \leqslant \mathit{b} \in \mathbb{R}_{+}\)\\
        So, \(\emptyset \in \mathit{S}\).\\
        Second, prove that \(\mathit{X} \subseteq \mathit{Y} \in \mathit{S} \Rightarrow \mathit{X} \in \mathit{S}\)\\
        \[\mathit{Y} \in \mathit{S} \Rightarrow \mathit{a}(\mathit{Y}) = \sum_{\mathit{j} \in \mathit{Y}}\mathit{a_j} \leqslant \mathit{b}\]
        \[\mathit{X} \subseteq \mathit{Y} \Rightarrow \mathit{a}(\mathit{X}) = \sum_{\mathit{j} \in \mathit{X}}\mathit{a_j} = \sum_{\mathit{i} \in \mathit{Y}}\mathit{a_i} - \sum_{\mathit{j} \in \mathit{Y} \setminus \mathit{X}}\mathit{a_j} \leqslant \sum_{\mathit{j} \in \mathit{Y}}\mathit{a_j} \leqslant \mathit{b}\]
        So, this difines an independence system \(\mathcal{F}\subseteq \mathrm{2}^{\mathit{E}}\)
    }
    \subsubsection*{b.}{
        According to the defination, we have,
        \[\mathrm{n=6} \Rightarrow \mathit{S} \subseteq \mathit{E} = \{ \mathrm{1,2,3,4,5,6}\}\]
        \[\mathit{a}=\mathrm{(1,1,1,4,4,5), b=8} \Rightarrow \mathit{a}(\mathit{S}) = \sum_{\mathit{j} \in \mathit{S}}\mathit{a_j} \leqslant \mathrm{8}\]
    }
}

\section*{Question 2}{

}

\section*{Question 3}{

}

\section*{Question 4}{

}

\section*{Question 5}{

}

\section*{Question 6}{

}

\section*{Question 7}{

}

\section*{Question 8}{

}

\section*{Question 9}{

}

\section*{Question 10}{
    \[\mathcal{B}+\mathcal{F}\]
}



\end{document}
